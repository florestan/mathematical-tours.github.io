\chapter*{Forewords}

There are many books on the Fourier transform; however, few are aimed at a multidisciplinary audience. Writing a book for engineers with algebraic concepts is a real challenge, as much, if not more, than writing an algebra book that puts the finger on the applications of the theories encountered. This is the challenge that this book has attempted to meet. Thus, each reader will be able to create a \guill{à la carte} program and draw from statements or computer programs specific elements to establish their knowledge in the field or apply them to more concrete problems.

The presentation is intentionally very detailed and requires little prior knowledge, mentioned at the beginning of the chapters concerned. 
% There is no doubt that a good student of license should be able to approach this presentation without great difficulty. 
The reader may occasionally need some advanced notions on finite groups as well as some familiarity with group actions. %An aggregative student should be able to find many applications and developments around the official program.
I did not hesitate to repeat the important definitions and notations. For example, the notion of convolution, approached from many angles (abelian group, signal processing, non-commutative group), is each time placed in its context. Thus, the different paragraphs, although following a logical progression, have real unity and can be read in a non-linear fashion.

The first chapter uses the language of group theory to explain the main concepts and demonstrate the statements that will be used later. The second chapter applies the results obtained to various problems and constitutes the first contact with fast algorithms (\textit{Walsh transform} for example). The third chapter is an exposition on the discrete Fourier transform. Even if it reinvests the results of the first chapter, it can be read for example by a computer scientist wishing to understand the mechanisms of the algorithms of discrete transforms. The fourth chapter presents various applications of the discrete Fourier transform, and constitutes an essential complement to the previous chapter, to fully understand the mechanisms involved as well as their use in practical situations. The fifth chapter presents more original ideas and algorithms around the Fourier transform, giving rise to numerous applications. The sixth chapter requires some more advanced knowledge, especially some familiarity with finite field theory. It studies valued transforms in a finite field and presents applications to corrective codes. The last two chapters (the most difficult), are of more algebraic nature and propose to generalize the constructions already carried out in the case of finite non-commutative groups. The seventh chapter exposes the theory of linear representations. The eighth and final chapter applies this theory in both theoretical (study of the simplicity of groups) and practical (spectral analysis) fields.

%A good knowledge of the algebraic properties of the Fourier transform is, in my opinion, very useful for constructing aggregation lessons that are both application-oriented and with solid theoretical bases. Many concepts in the aggregation program will be reviewed in this book. First of all, the notion of finite groups is at the heart of the problem approached in this book. Cyclic groups such as $\ZZ / n \ZZ $ are more particularly highlighted: they are the simplest groups, but also the most used in practice. The complex numbers of module 1 are present throughout the talk. The use of Hermitian spaces and unitary transformations is a constant in Fourier theory. The continuous Fourier transform and the Fourier series will be approached only in the exercises, however, the discrete Fourier transform makes it possible to enrich their analysis. Solving partial differential equations makes full use of the algebraic properties of Fourier transforms. Moreover, the calculation of the Fourier coefficients by the fast Fourier transform algorithm, as well as the considerations on the convolution during the resolution by finite differences, make the discrete Fourier transform an essential tool. Finally, the theory of finite fields can also be approached through the world of Fourier.


\index{Matlab} \index{Maple} A number of computer programs are presented; they are written in \textsc{Matlab} for the most part, and in \textsc{Maple} for those which require algebraic manipulations (calculations in finite fields, etc.). Although these are commercial software, free software with very similar syntax exists, such as  \textsc{Scilab}, \textsc{Octave} and \textsc{Mupad}. 
% 
The choice of a particular language to implement the algorithms is obviously debatable, but the use of \textsc{Matlab} and \textsc{Maple} seems quite natural, these software allow us to quickly test the written programs. We can then translate them into a compiled and faster language, such as C or C++. 
%
% In addition, these languages are used for the oral modeling test of the mathematics aggregation. 
%
% The aggregates or future aggregates will therefore not be disoriented. It should be noted that all the programs in this book are available for download, as well as many others, on the \url{https://adtf-livre.github.io/} site.


% I would like to thank my parents, Lucien and Marie-No\"{e}lle, who supported me throughout the writing of this book. Finally, I address my deepest gratitude to my proofreaders, who provided their experience to point me in the right direction: Abdellah Bechata, Vincent Beck, Christophe Bertault, Jérôme Malick and Jean Starynkévitch.

% \vskip 3mm \begin{minipage}[c]{.8\linewidth} \end{minipage}\hfill \begin{minipage}[c]{.2\linewidth} Gabriel Peyré. \end{minipage}\newpage 
%  \thispagestyle{empty} 